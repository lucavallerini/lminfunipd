\documentclass[a4paper]{scrartcl}
\usepackage[a4paper,top=2cm,bottom=2cm,left=2cm,right=2cm]{geometry}
\usepackage[utf8x]{inputenc}
\usepackage{graphicx}
\usepackage[T1]{fontenc}
\usepackage[italian]{babel}
\usepackage{tabularx}
\usepackage{helvet}
\usepackage{longtable}
\usepackage{booktabs}
\usepackage{xcolor}
\usepackage{amsfonts}
\renewcommand{\familydefault}{\sfdefault}

% Informazioni generali
\newcommand{\corso}{Basi di Dati}
\newcommand{\anac}{A.A. 2014/2015}
\newcommand{\laurea}{Corso di Laurea Magistrale in Ingegneria Informatica}
\newcommand{\matricola}{1110975}
\newcommand{\nome}{Luca}
\newcommand{\cognome}{Vallerini}
\newcommand{\data}{Data di consegna: 30 aprile 2015}
\newcommand{\consegna}{Homework 2 - Progettazione Concettuale}

\begin{document}
\begin{footnotesize}

% Intestazione con loghi
\begin{figure}
	\begin{minipage}[t]{\textwidth}
		\includegraphics[height=15mm]{img/logounipd}
		\hfill
		\includegraphics[height=15mm]{img/logodei}
	\end{minipage}
\end{figure}

% Intestazione corso
{
\centering
\textbf{\corso , \anac} \\
\textbf{\laurea} \\
\vspace{5pt}
\textbf{\consegna} \\
\textbf{\small\data}


% Intestazione dati studente
\begin{table}[h]
	\begin{tabularx}{\textwidth}{|X|X|X|}
		\hline
		\multicolumn{1}{|c|}{\textbf{Cognome}} &
		\multicolumn{1}{c|}{\textbf{Nome}} &
		\multicolumn{1}{c|}{\textbf{Numero di matricola}} \\
		\centering\cognome &
		\centering\nome &
		\centering\matricola \tabularnewline
		\hline
	\end{tabularx}
\end{table}

}	

% Svolgimento
\subsection*{\color[RGB]{155,0,20}Schema Entità-Associazione}
È possibile trovare lo schema concettuale in fondo al presente documento.

\subsection*{\color[RGB]{155,0,20}Dizionario dei dati}
% ENTITÀ - DESCRIZIONE - ATTRIBUTI - IDENTIFICATORI
\begin{longtable}{p{0.2\columnwidth} p{0.3\columnwidth} p{0.2\columnwidth} p{0.2\columnwidth}}
	\toprule
	\multicolumn{1}{l}{\textbf{ENTITÀ}} &
	\multicolumn{1}{l}{\textbf{DESCRIZIONE}} &
	\multicolumn{1}{l}{\textbf{ATTRIBUTI}} &
	\multicolumn{1}{l}{\textbf{IDENTIFICATORI}} \\
	\midrule
	\endfirsthead
	\multicolumn{4}{l}{\textit{\footnotesize continua dalla pagina precedente}} \\
	\toprule
	\multicolumn{1}{l}{\textbf{ENTITÀ}} &
	\multicolumn{1}{l}{\textbf{DESCRIZIONE}} &
	\multicolumn{1}{l}{\textbf{ATTRIBUTI}}  &
	\multicolumn{1}{l}{\textbf{IDENTIFICATORI}} \\
	\midrule
	\endhead
	\midrule
	\multicolumn{4}{r}{\textit{\footnotesize continua nella pagina successiva}} \\
	\endfoot
	\bottomrule
	\endlastfoot
	
	% A
	ABBONAMENTO &
	Ordine che permette ad un cliente di abbonarsi ad un prodotto di tipo EDIZIONE. L'ordine può essere inserito direttamente dal cliente o tramite un membro dello staff per conto del cliente. L'ordine si conclude con l'inserimento dei prodotti richiesti nella casella del cliente &
	- &
	- \\
	
	ACCOUNT &
	Sezione del sito che permette al proprietario dell'account di usufruire dei servizi di cui ha diritto in base al proprio ruolo. Genitore di ACCOUNT\_{CLIENTE} e ACCOUNT\_{STAFF}  &
	CodAcc, PW, Stato, User &
	CodAcc \\
	
	ACCOUNT\_{CLIENTE} &
	Account dedicato alla sola figura del cliente. Figlia di ACCOUNT &
	- &
	- \\	
	
	ACCOUNT\_{STAFF} &
	Account dedicato alla sola figura dello staff. Figlia di ACCOUNT &
	Titolare &
	- \\
	
	ACQUISTO &
	Vendita di prodotti al cliente finale. L'ordine viene inserito e direttamente lavorato da un membro dello staff. Se il cliente che effettua l'acquisto è un cliente tesserato esso viene collegato a tale ordine. Figlia di ORDINE\_{FUMETTERIA} &
	- &
	- \\
	
	ANIME &
	Identifica una serie animata giapponese con le sue caratteristiche peculiari. Figlia di SERIE &
	\#{Ep}, Regista, Studio &
	- \\
	
	ARRETRATO &
	Ordine che permette ad un cliente di recuperare un prodotto di tipo VOLUME (sia esso già pubblicato o da pubblicare). L'ordine può essere inserito direttamente dal cliente o tramite un membro dello staff per conto del cliente. L'ordine si conclude con l'inserimento dei prodotti richiesti nella casella del cliente. Figlia di ORDINE\_{CLIENTE}  &
	- &
	- \\
	
	% C
	CASELLA &
	Luogo fisico e virtuale nel quale vengono tenuti da parte dei prodotti richiesti dal cliente titolare della casella per un fissato periodo di tempo &
	CodCas &
	CodCas \\

	% D
	DISTRIBUTORE & 
	Azienda che si occupa della distribuzione dei prodotti editi da una casa editrice &
	Fax, Mail, Nome, P.IVA, Sito, Tel &
	P.IVA \\
	
	% E
	EDITORE &
	Azienda che si occupa della pubblicazione dell'edizione italiana di una SERIE &
	Fax, Mail, Nome, P.IVA, Sito, Tel &
	P.IVA \\
	
	EDIZIONE &
	Insieme delle caratteristiche editoriali comuni a tutti i volumi che compongono una pubblicazione italiana di una SERIE. Figlia di PRODOTTO. Genitore di HOMEVIDEO e FUMETTO &
	\#{VolTot}, Extra, InizPubbl, Period, Stato &
	- \\
	
	% F
	FORNITURA &
	Ordine che permette alla fumetteria di rifornirsi dei prodotti da mettere in vendita. La richiesta vera e propria viene effettuata al DISTRIBUTORE &
	- &
	- \\
	
	FUMETTO &
	Identifica le caratteristiche peculiari della pubblicazione italiana di un manga. Figlia di EDIZIONE &
	\#{Pag}, Col, SovraCop &
	- \\
	
	% H
	HOMEVIDEO &
	Identifica le caratteristiche peculiari della pubblicazione italiana di una serie animata in DVD o BD. Figlia di EDIZIONE &
	Audio, Sub, Supporto, Video &
	- \\
	
	% M
	MAIL &
	Identifica un messaggio di notifica. Può essere inviato come una vera e propria email &
	CodMail, DataInvio, Ogg, Text &
	CodMail \\
	
	MANGA &
	Identifica una serie a fumetti giapponese con le sue caratteristiche specifiche. Figlia di SERIE  &
	\#{VolTot}, Rivista &
	- \\
	
	% O
	ORDINE &
	Categoria generale che racchiude i vari tipi di ordini che è possibile effettuare. Genitore di ORDINE\_{CLIENTE} e ORDINE\_{FUMETTERIA}. L'ordine viene sempre lavorato da un membro dello staff &
	CodOrd, DataIns, PrezzoTot, Stato &
	CodOrd, Stato \\
	
	ORDINE\_{CLIENTE} &
	Particolare tipo di ORDINE che può essere inserito sia dal cliente che da un membro dello staff per conto di un cliente. Tale ordine può essere annullato dal cliente se e solo se non è già entrato in lavorazione. Figlia di ORDINE. Genitore di ABBONAMENTO e ARRETRATO &
	- &
	- \\	
	
	ORDINE\_{FUMETTERIA} &
	Particolare tipo di ORDINE che può essere inserito solamente da un membro dello staff. Figlia di ORDINE. Genitore di ACQUISTO e FORNITURA &
	- &
	- \\
	
	% P
	PRODOTTO &
	Singola entità che è in vendita. &
	CodProd, DataIns &
	CodProd \\
	
	% S
	SERIE &
	Rappresenta un ANIME o un MANGA e le loro caratteristiche comuni. Genitore di ANIME e MANGA &
	Anno, Autore, Genere, Stato, Titolo, Trama &
	Anno, Autore, Titolo \\
	
	% T
	TESSERA &
	Oggetto fisico che garantisce servizi aggiuntivi ad un cliente &
	CodTess, DurataCasella, FineVal, InizioVal, Prezzo, Sconto &
	CodTess \\
	
	TESSERATO &
	Cliente che vuole tesserarsi per usufruire di servizi aggiuntivi &
	CF, Cognome, Indirizzo (Via, CAP, Città, Provincia), Mail, Nascita, Nome, Tel
	CF \\
	
	% V
	VOLUME &
	Singolo elemento di una pubblicazione seriale di un'EDIZIONE di un manga o un anime &
	\#{Rist}, \#{Vol}, DataPubbl, ISBN, Prezzo, Ristampa &
	ISBN \\
	
\end{longtable}

% ASSOCIAZIONE - DESCRIZIONE - COMPONENTI - ATTRIBUTI
\begin{longtable}{p{0.2\columnwidth} p{0.4\columnwidth} p{0.2\columnwidth} p{0.1\columnwidth}}
	\toprule
	\multicolumn{1}{l}{\textbf{ASSOCIAZIONE}} &
	\multicolumn{1}{l}{\textbf{DESCRIZIONE}} &
	\multicolumn{1}{l}{\textbf{COMPONENTI}} &
	\multicolumn{1}{l}{\textbf{ATTRIBUTI}} \\
	\midrule
	\endfirsthead
	\multicolumn{4}{l}{\textit{\footnotesize continua dalla pagina precedente}} \\
	\toprule
	\multicolumn{1}{l}{\textbf{ASSOCIAZIONE}} &
	\multicolumn{1}{l}{\textbf{DESCRIZIONE}} &
	\multicolumn{1}{l}{\textbf{COMPONENTI}} &
	\multicolumn{1}{l}{\textbf{ATTRIBUTI}} \\
	\midrule
	\endhead
	\midrule
	\multicolumn{4}{r}{\textit{\footnotesize continua nella pagina successiva}} \\
	\endfoot
	\bottomrule
	\endlastfoot
	
	% A
	acquirente &
	Colui che acquista un prodotto (solo se TESSERATO) &
	ACQUISTO, TESSERATO &
	- \\
	
	articolazione &
	Lega una particolare EDIZIONE di una SERIE e i vari VOLUMI che la compongono & 
	EDIZIONE, VOLUME &
	- \\
	
	associata &
	Legame tra un ACCOUNT\_{CLIENTE} e la CASELLA ad esso designata &
	ACCOUNT\_{CLIENTE}, CASELLA &
	- \\
	
	% C
	coinvolge &
	La casella (e quindi il cliente tesserato) che beneficia di un ORDINE\_{CLIENTE} (necessario specificarlo poiché può essere un membro dello staff che crea l'ordine per il cliente) &
	CASELLA, ORDINE\_{CLIENTE} &
	- \\
	
	% D
	distribuzione &
	Lega un EDITORE e l'azienda che si fa carico della distribuzione dei suoi prodotti &
	DISTRIBUTORE, EDITORE &
	- \\
	
	% I
	incasellamento &
	Inserimento o rimozione in casella di uno o più prodotti richiesti tramite un ORDINE\_{TESSERATO} &
	ACCOUNT\_{STAFF}, CASELLA, PRODOTTO &
	DataIns, DataRim, DataScad, Quantità \\
	
	inserimento &
	Creazione di un ordine &
	ACCOUNT, ORDINE &
	Data \\
	
	% L
	lavorazione &
	Rappresenta le varie fasi della lavorazione di un ordine &
	ACCOUNT\_{STAFF}, ORDINE &
	Data \\
	
	% M
	mailbox &
	Legame tra il destinatario e il mittente di una MAIL  &
	ACCOUNT, MAIL &
	Data \\
	
	merce &
	PRODOTTO che viene inserito in un ORDINE &
	PRODOTTO, ORDINE &
	Quantità \\
	
	% P
	pubblicazione &
	Pubblicazione dell'EDIZIONE italiana di una SERIE da parte di un EDITORE &
	EDITORE, EDIZIONE &
	- \\
	
	% R
	riguarda &
	La SERIE oggetto della pubblicazione italiana &
	EDIZIONE, SERIE &
	- \\
	
	rifornimento &
	Richiesta ad un distributore di un prodotto di cui la fumetteria vuole rifornirsi &
	- \\	
	
	% T
	tesseramento &
	Atto con il quale un cliente ottiene una tessera e i servizi aggiuntivi ad essa collegati &
	ACCOUNT\_{CLIENTE}, ACCOUNT\_{STAFF}, TESSERA, TESSERATO &
	Data \\
	
	tratto &
	Descrive un legame tra una serie ANIME e una serie MANGA. Es.: l'anime DRAGONBALL è tratto dal MANGA omonimo &
	ANIME, MANGA &
	- \\
	
\end{longtable}

% ATTRIBUTO - ENTITÀ/ASSOCIAZIONE - DOMINIO - DESCRIZIONE
\begin{longtable}{p{0.1\columnwidth} p{0.2\columnwidth} p{0.3\columnwidth} p{0.3\columnwidth}}
	\toprule
	\multicolumn{1}{l}{\textbf{ATTRIBUTO}} &
	\multicolumn{1}{l}{\textbf{COMPONENTE}} &
	\multicolumn{1}{l}{\textbf{DOMINIO}} &
	\multicolumn{1}{l}{\textbf{DESCRIZIONE}} \\
	\midrule
	\endfirsthead
	\multicolumn{4}{l}{\textit{\footnotesize continua dalla pagina precedente}} \\
	\toprule
	\multicolumn{1}{l}{\textbf{ATTRIBUTO}} &
	\multicolumn{1}{l}{\textbf{ENTITÀ/ASSOCIAZIONE}} &
	\multicolumn{1}{l}{\textbf{DOMINIO}} &
	\multicolumn{1}{l}{\textbf{DESCRIZIONE}} \\
	\midrule
	\endhead
	\midrule
	\multicolumn{4}{r}{\textit{\footnotesize continua nella pagina successiva}} \\
	\endfoot
	\bottomrule
	\endlastfoot
	
	% #
	\#{Ep} &
	ANIME &
	int & 
	Numero di episodi di cui è composta una serie animata \\
	
	\#{Pag} &
	FUMETTO &
	int &
	Numero medio di pagine di cui è composto un volume cartaceo \\
	
	\#{Rist} &
	VOLUME &
	int &
	Numero della ristampa \\
	
	\#{Vol} &
	VOLUME &
	int &
	Numero del volume all'interno di una serie composta da più volumi \\
	
	\#{VolOrig} &
	MANGA &
	int &
	Numero totale di volumi dell'edizione originale \\
	
	\#{VolTot} &
	EDIZIONE &
	int &
	Numero totale di volumi di cui è composta un'edizione italiana \\
	
	% A
	Anno &
	SERIE &
	int &
	Anno di inizio pubblicazione dell'edizione giapponese \\
	
	Audio &
	HOMEVIDEO &
	text &
	Caratteristiche del comparto audio di un'edizione italiana di una serie animata \\
	
	Autore &
	SERIE &
	text &
	Autore originale di un anime o di un manga \\
	
	% C
	CAP &
	TESSERATO &
	text &
	Codice di avviamento postale dell'indirizzo di un cliente \\
	
	CF &
	TESSERATO &
	text &
	Codice fiscale del cliente \\
	
	Città &
	TESSERATO &
	text &
	Città dell'indirizzo di un cliente \\
	
	Civico &
	TESSERATO &
	text &
	Numero civico dell'indirizzo di un cliente \\
	
	CodAcc &
	ACCOUNT &
	text &
	Codice che identifica univocamente un account \\
	
	CodCas &
	CASELLA &
	text &
	Codice che identifica univocamente una casella \\
	
	CodMail &
	MAIL &
	text &
	Codice che identifica univocamente un messaggio di notifica \\
	
	CodOrd &
	ORDINE &
	text &
	Codice che identifica univocamente un ordine \\
	
	CodProd &
	PRODOTTO &
	text &
	Codice che identifica univocamente un prodotto \\
	
	CodTess &
	TESSERA &
	text &
	Codice che identifica univocamente una tessera \\
	
	Cognome &
	TESSERATO &
	text &
	Cognome del cliente \\
	
	Col &
	FUMETTO &
	boolean &
	Vero se l'edizione contiene pagine a colori, falso altrimenti \\
	
	% D
	Data & 
	inserimento, lavorazione, mailbox, tesseramento &
	timestamp &
	Istante in cui è avvenuta un'operazione \\
	
	DataIns &
	incasellamento, ORDINE, PRODOTTO &
	timestamp &
	Data di inserimento \\
	
	DataInvio &
	MAIL &
	timestamp &
	Istante dell'invio della notifica \\
	
	DataPubbl &
	VOLUME &
	date &
	Giorno di pubblicazione di un volume \\
	
	DataRim &
	incasellamento &
	date &
	Data di rimozione di un volume dalla casella \\
	
	DataScad &
	incasellamento &
	date &
	Data di termine della giacenza di un prodotto in casella \\
	
	Disp &
	VOLUME &
	int &
	Numero di volumi di uno stesso numero disponibili in fumetteria \\
	
	DurataCasella &
	TESSERA &
	int &
	Stabilisce la durata massima (in mesi) per le giacenze in casella \\
	
	% E
	Extra &
	EDIZIONE &
	text &
	Altre informazioni sull'edizione \\
	
	% F
	Fax &
	DISTRIBUTORE, EDITORE &
	text &
	Numero di fax \\
	
	FineVal &
	TESSERA &
	date &
	Giorno di fine validità della tessera \\
	
	% G
	Genere &
	SERIE &
	text &
	Genere di appartenenza di una serie \\	

	Indirizzo &
	TESSERATO &
	- &
	Indirizzo (Via, Civico, CAP, Città) di un cliente \\
	
	InizioVal &
	TESSERA &
	date &
	Giorno di inizio di validità della tessera \\	
	
	InizPubbl &
	EDIZIONE &
	date &
	Data di inizio della pubblicazione di un'edizione \\
		
	ISBN & 
	VOLUME &
	text &
	Codice univoco mondiale che identifica un determinato volume \\
		
	% M
	Mail &
	ACCOUNT &
	text &
	Indirizzo email di un cliente \\
	
	% N
	Nascita &
	TESSERATO &
	date &
	Data di nascita del cliente \\
	
	Nome &
	DISTRIBUTORE, EDITORE, TESSERATO &
	text &
	Nome proprio o di azienda \\
	
	% O
	Ogg &
	MAIL &
	text &
	Oggetto della notifica \\
	
	% P
	Period &
	EDIZIONE &
	text &
	Periodicità della pubblicazione italiana \\
	
	P.IVA &
	DISTRIBUTORE, EDITORE &
	text &
	Partita IVA aziendale \\
	
	PrezzoFin &
	VOLUME &
	money &
	Prezzo di vendita di un volume \\
	
	PrezzoTess &
	TESSERA &
	money &
	Prezzo della tessera \\
	
	PrezzoTot &
	ORDINE &
	money &
	Conto totale di un ordine \\
	
	Provincia &
	TESSERATO &
	text &
	Sigla della provincia dell'indirizzo di un cliente \\
	
	PW &
	ACCOUNT &
	text &
	Password di accesso ad un account \\
	
	% Q
	Quantità &
	INCASELLAMENTO, MERCE &
	int &
	Quantità di un prodotto \\
	
	% R
	Regista &
	ANIME &
	text &
	Regista della serie animata \\
	
	Ristampa &
	VOLUME &
	boolean &
	Vero se è una ristampa, falso se è una prima edizione \\
	
	Rivista &
	MANGA &
	text &
	Luogo di pubblicazione dell'edizione originale di un manga \\
	
	% S
	Sconto &
	TESSERA &
	int &
	Sconto percentuale su ogni acquisto effettuato \\
	
	Sito &
	DISTRIBUTORE, EDITORE &
	text &
	Sito internet \\
	
	Studio &
	ANIME &
	text &
	Studio di produzione di un anime \\
	
	Sub &
	HOMEVIDEO &
	text &
	Caratteristiche dei sottotitoli di un'edizione italiana di un anime \\
	
	SovraCop &
	FUMETTO &
	boolean &
	Vero se l'edizione presenta la sovraccoperta, falso altrimenti \\
	
	StatoAcc &
	ACCOUNT &
	enum (attivo, sospeso, disattivo) &
	Stato di un account \\
	
	StatoEd &
	EDIZIONE &
	enum (in corso, sospesa, interrotta, terminata) &
	Stato della pubblicazione di un'edizione \\
	
	StatoOrd &
	ORDINE &
	enum (nuovo, lavorazione, evaso parziale, evaso, annullato)  &
	Stato di un ordine \\
	
	StatoSerie &
	SERIE &
	enum (in corso, sospesa, interrotta, terminata) &
	Stato di una serie in madrepatria \\
	
	Supporto &
	HOMEVIDEO &
	enum (DVD, BD) &
	Indica il tipo di supporto fisico sul quale è pubblicata l'edizione italiana di un anime \\
	
	% T
	Tel &
	DISTRIBUTORE, EDITORE, TESSERATO &
	text &
	Numero di telefono \\
	
	Text &
	MAIL &
	text &
	Testo della notifica \\	

	Titolare &
	ACCOUNT\_{STAFF} &
	boolean &
	Vero se è l'account di un proprietario, falso se è l'account di un venditore \\	
	
	Titolo & 
	SERIE &
	text &
	Titolo di una serie animata o cartacea \\
	
	Trama &
	SERIE &
	text &
	Breve sinossi della serie \\
	
	% U
	User &
	ACCOUNT &
	text &
	Nome utente di un account \\
	
	% V
	Via &
	TESSERATO &
	text &
	Via dell'indirizzo di un cliente \\
	
	Video &
	HOMEVIDEO &
	text &
	Caratteristiche del comparto video di un'edizione italiana di un anime \\
	
\end{longtable}

\subsection*{\color[RGB]{155,0,20}Vincoli esterni}
\begin{itemize}
\item \textbf{Calcolo dei prezzi} Per tutti gli ORDINI eccetto FORNITURA il calcolo del prezzo totale dell'ordine viene effettuato come la somma del \textit{PrezzoFin} dei singoli prodotti; se l'ordine inoltre riguarda un cliente tesserato il \textit{PrezzoTot} corrisponde al prezzo scontato in base allo \textit{Sconto}. Per l'ordine di fornitura viene invece inserito manualmente in base al prezzo stipulato di volta in volta con il distributore;

\item \textbf{ORDINE\_{FUMETTERIA}} Gli ordini di questo tipo possono essere creati \textbf{solo ed esclusivamente} da un membro dello staff. In particolare l'ordine FORNITURA può essere creato \textbf{solo} da un proprietario (attributo \textit{Titolare} posto a true);

\item \textbf{ACQUISTO} Se l'acquisto viene effettuato da un cliente tesserato esso \textbf{deve} entrare in partecipazione nell'associazione \textit{acquirente}, altrimenti \textbf{non deve} parteciparvi;

\item Per una più semplice gestione gli identificatori di ACCOUNT\_{CLIENTE}, TESSERA e CASELLA devono coincidere se associati per un medesimo TESSERATO.
\end{itemize}


\subsection*{\color[RGB]{155,0,20}Verifica requisiti funzionali}
\begin{itemize}
\item \textbf{Catalogo} Tramite le entità VOLUME, EDIZIONE e SERIE e le relative associazioni che le mettono in relazione è possibile consultare l'intero catalogo dei prodotti, visualizzare le caratteristiche dei prodotti ed effettuare le modifiche richieste. Se si estendono le interrogazioni agli ordini effettuati è anche possibile estrarre opportune statistiche;

\item \textbf{Ordini e vendite} La rappresentazione fornita permette l'inserimento e la lavorazione degli ordini sfruttando le associazioni che legano ORDINI e ACCOUNT. Con le opportune interrogazioni è possibile risalire allo storico degli ordini e in particolare di quello degli acquisti;

\item \textbf{Tesseramento} L'associazione \textit{tesseramento} permette di gestire i clienti che vogliono usufruire di questa possibilità: in particolare è possibile per un membro dello staff tesserare un nuovo cliente registrandolo con i suoi dati anagrafici e di contatto nonché creando i servizi aggiuntivi rappresentati dalle entità TESSERA, CASELLA, ACCOUNT\_{CLIENTE} e MAIL;

\item \textbf{Profilo venditore e profilo proprietario} Gli account dei membri dello staff in particolare possono gestire gli ordini (creazione e lavorazione), sono in grado di inserire o rimuovere i prodotti nelle caselle e hanno la possibilità di tesserare i clienti;

\item \textbf{Stampa} Dai punti precedenti segue che il requisito sulla stampa delle ricerche e delle statistiche è soddisfatto.
\end{itemize}


\subsection*{\color[RGB]{155,0,20}Modifiche all'analisi dei requisiti}
Non sono avvenute grosse modifiche rispetto all'analisi dei requisti effettuata. Data la natura del progetto si è deciso di non implementare una forma di controllo totale sulle operazioni eseguite dagli utenti che hanno accesso al sistema.

\end{footnotesize}
\end{document}